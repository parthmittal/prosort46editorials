\section{Working of an online judge}
    \subsection{Terms}
        \begin{center}
            \begin{tabular}{| l | p{28em} |}
                \hline
                Constraints & The test data will obey any conditions imposed in the constraints. Eg. $N \leq 100$ means that $N \leq 100$ in all test cases. \\
                \hline
                Memory limit & The amount of RAM your program is allowed to use. \\
                \hline
                Time limit & The amount of CPU time your program gets. \\
                \hline
            \end{tabular}
        \end{center}
    \subsection{Running your code}
        \begin{itemize}
            \item
                Your code is compiled, eg. \verb|g++ -O2 ramesh.cpp -o ramesh|.
            \item
                Your code is then run on the first test case,\\eg. \verb|./ramesh < firstTestCase.in > firstTestCaseParticipant.out|.
            \item
                Your output for the first test case is then compared with the jury's output for the first test case.
            \item
                Your code is then run on the second test case,\\ eg. \verb|./ramesh < secondTestCase.in > secondTestCaseParticipant.out|.
            \item
                Your output for the second test case is then compared with the jury's output for the second test case.
            \item
                Your code is then run on the third test case,\\ eg. \verb|./ramesh < thirdTestCase.in > thirdTestCaseParticipant.out|.
            \item
                Your output for the third test case is then compared with the jury's output for the third test case.
            \item[] \begin{center} \textbf{.} \end{center}
            \item[] \begin{center} \textbf{.} \end{center}
            \item[] \begin{center} \textbf{.} \end{center}
            \item
                Your code is then run on the last test case,\\ eg. \verb|./ramesh < lastTestCase.in > lastTestCaseParticipant.out|.
            \item
                Your output for the last test case is then compared with the jury's output for the last test case.
        \end{itemize}
    \subsection{Verdicts}
        \begin{center}
            \begin{tabular}{| l | p{25em} |}
                \hline
                Compile Error & Your code failed to compile. \\
                \hline
                Wrong Answer & For at least one test case, the output your program produced did not match the problem specifications. Note that this test case may be different from the sample test case, so arguments like ``But it passes on the sample test case'' are stupid and void. \\
                \hline
                RTE & For at least one test case, your program terminated in an unusual fashion. This can be caused by accessing memory you haven't reserved, a function call stack overflowing, or any number of things. Organizers generally cannot figure out why you have a run-time error. \\
                \hline
                TLE & For at least one test case, your program takes more time than the Time Limit. Note that this does not mean your program produces correct output, since it got terminated before it completed. Further, note that the test case in question may be different from the sample test case, so arguments like ``But it passes on the sample test case'' are stupid and void. \\
                \hline
                MLE & For at least one test case, your program takes more memory than the Memory Limit. Note that this does not mean your program produces correct output, since it got terminated before it completed. \\
                \hline
                Judgement Failure & Our judge is not familiar with this verdict, since it never happens to us. Maybe you should ask the guys over at \textcolor{blue}{\href{https://www.codechef.com}{CodeChef}}. \\
                \hline
            \end{tabular}
        \end{center}
    As I am sure you appreciate now, an online judge is very complicated. Please make a few submissions at your friendly neighbourhood online judge, to familiarise yourself with its workings.
