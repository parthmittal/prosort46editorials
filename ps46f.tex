\section{Problem F}
	\subsection{Statement}
		\href{http://foobar.iiitd.edu.in/contest/team/problem.php?id=532}{click}

	\subsection{Solution}
		The key idea of this problem is to consider the process backwards. If we fix some very far moment of time (i.e $10^{50}$, as in the problem statement), and now look operation in the reverse direction, the problem becomes the following: a ball is chosen with probability $p_i$ and if it is not present in the current set, we add it there. We stop if there are $k$ elements in the set. What is the probability the $i$-th element will be present in the set at the end? 

		The intuition behind that is, consider a long time to have passed. Which balls are in the tower? The last $k$ distinct balls which were picked will be in the tower. So going backwards, we pick a ball. If it is already in the tower, we can ignore it since we're looking for the last $k$ \emph{distinct} balls. We keep picking a ball until we have $k$ elements in the tower, and then we stop.

		This is now much easier to solve, as we should only add new elements, without removing anything. As a result, we do not have to store the order elements were added and are only interested in the set. 

		Calculate the following dynamic programming: $dp[mask]$ = the probability to have the set given by some $mask$ at some point of the  process. We can move from this state to all $new_{mask}$ that are equal to $mask + (1 << i)$ for all $i$ not present in $mask$.

		Now, to obtain the answer for some particular element we just have to pick all $mask$ of size $k$ that contain this element.

		One detail we have to deal with is that zero probability $p_i = 0$ is allowed. Remove all such elements from the set and reduce the value of $k$ if $n$ becomes less than $k$.